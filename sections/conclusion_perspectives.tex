
%%%% OK

In this report, we presented a data dissemination approach that disseminates data through users of a crowd-sensing platform.
This approach is based on a notion of proximity between users exchange data--proximity that could be Physical, Social or Contextual.
We provide an Android library called Foug\`ere that includes all the concepts presented in this report. 
In addition, we provide a large-scale emulation platform called AndroFleet, which is useful to test--as near as possible of a real usage--Android applications in the context of crowd-sensing.
\\
% Fougère and AndroFleet perspectives

As perspectives, different threat models could be tested on applications using Foug\`ere to see in which cases our approach is efficient and improve it in the cases where it fails.
Foug\`ere could take concern of what type of data is sent and auto adapt the parameters (e.g., the TTL value) in function of the data flow monitored.
The Wi-Fi Direct could implement a new method base on decentralized k-anonymity where a data would be sent to the server only if k instances of this data have been received.
Finally, we could include Foug\`ere in the APISENSE Android application to test his viability in the real world.
\\

Concerning AndroFleet emulator, it could be improved to cover a larger experimentation cases.
We could enrich the implementation of the Wi-Fi Direct simulator for a better and larger support of the technology.
Additionally, the Bluetooth technology could be simulated to offer more a proximity networking alternative.
At last, localized Wi-Fi hotspot could be useful for some experimentations.
\\
% Evaluation perspectives

Future work to become is to run the evaluation protocol described in this report and interpret the results.
Foug\`ere library and AndroFleet are functional and to be able to conduct the results, the only requirement is to well configured each part and run it on the testing server.
\\

Last, but not least, this work will be submitted--at the end of the month of September--to the famous IEEE PerCom conference. 

%%%% END OK
