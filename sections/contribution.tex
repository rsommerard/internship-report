



A FougereModule must have a ratio value.
This ratio value is used by Foug\`ere when it allocate data to the different modules.
The sum of module ratios activated cannot be up to 100.
Default module ratios are 60 for the WiFiDirect module, 30 for the Social module and 10 for the Contextual module -- 60\% of total data to the WiFiDirect module.
These ratios can be dynamically adapted during the application runtime.


Social and Contextual uses legacy technologies to disseminate data.
Uses 3G or Wi-Fi to disseminate these last.



We develop a library called \textit{Foug\`ere}\footnote{Foug\`ere is a french word that translate to "fern" in english. This plant symbolize the trust.}.
Foug\`ere come with a new principle of distance calculation to auto-adapt the data dissemination.

\subsection{Neighboring Sampling}

Foug\`ere uses a principle of neighbor.
The library can have various types of neighbors.
Our implementation includes 3 different types of neighbors: \textbf{Physical} neighbors (Wi-Fi Direct, Bluetooth), \textbf{Social} neighbors (Familly, Friends, colleagues) and \textbf{Contextual} neighbors (other participants to the same experiment).

We want to be sure that the data will be propagated.
We assume that the server don't know what are the Social relation that exist between different users.
Trusted social peers are only known by the user.

30\% of data are sended by the Social


Explain Contextual -> data collect campaign.

\subsection{Distance Evaluation}

Data to disseminate are allocate to the different types available.
The allocation is made in function of each type score.
Type score are regularly updated in function of the previous dissemination result.

For instance, if the user meet very rarely other devices, the score is adapted and a lower amount of data will be allocated to the Wi-Fi Direct.

We can choose, for a better privacy, to select what part enabled. For instance, we can enable only dissemination by physical way.

A score is calculated to allocate data that will be disseminated.

At the allocation phase, for each data we choose a random number with the probability relative to each type score. 

Each time a data is sended with success, the score for the specific type is incremented.

\subsection{Probabilist Data Dissemination}



A Time to Live (TTL) value is define for each data.
This TTL warrants a minimum number of transmission before the data is reported to the server.

The data are encrypted with an asymmetric encryption by using the server public key to prevent from curious devices. Only the server can read the data and see his content.

Each data has a dissemination value.
This value is used to determine the number of sending the data will be.

60\% of data go in WiFi-Direct pool, 30\% in Social pool and 10\% in the Contextal pool.
Why ? because we divide the dissemination. Without this repartition, if no data are sended by WiFi-Direct, we avoid the fact that all data are disseminate through a single road.
We let a chance that data are realocated to WiFi-Direct and disseminated through that channel.

We send the same data to, at least, 3 different neighbors.

At the startup, each neighbor has a chance of 0.5 to send a data to it. Each time an interaction is made with a neighbor, the chance to send it a data decrease. If no interaction is decided, we increase the chance for the next time.

A notion of history is include to avoid multiple exchange with the same neighbor in a time window.

When a data is received, it is store in the main database and will be schedule to different module at an other time.

Data that are sent to another peer as keep in the current bucket and will be re-allocated the next time.