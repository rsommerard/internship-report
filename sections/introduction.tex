
%%%% OK

% Basic context

A lot of attention is turn on the crowd-sensing platforms that allow to collect and aggregate knowledge from a large number of devices populating our world.
APISENSE~\cite{DBLP:conf/dais/HadererRS13} is a participatory platform that provides to scientists an easy way to deploy their sensing experiments in the wild.
This platform allows end-users to participate to data collection campaigns.
The data collected by this platform can be used by scientists that are able to assess their theories on real datasets instead of simulated ones.
\\
% Problem description

In APISENSE, scientists create data collect campaigns that gather data from the users' devices that have chosen to participate--data recovery is made via the APISENSE Android application installed on the users' devices.
Once a user agrees to participate in a specific campaign, the collecting process is completely automatic and no further interaction is required to the user sending data to a collect server.
The collect servers can be the central server or server managed by external organizations.
All external servers send their collected data to the central server.
To sum up, we have a system where users send their own produced data directly to a central server or to an external one that relay to the central server.
In this case, a collect server knows to whom is each data collected.
Despite the fact that the central server can be 100\% trusted by the users, a serious issue regarding the users' privacy may appear if this server is hacked.
Furthermore, the servers maintain by organization provide no guarantee on their use of data before forward their data to the central server.
A malicious server could store each collected data before forwarding it to the central server.
This treat will be totally transparent and undetectable by the central server.
\\
% Deficiencies with previous approaches

The APISENSE platform implements some privacy protecting methods on the server side and user side.
For instance, on the server side, different filters can be enabled by a scientist to enforce the users' privacy, but filter configurations remain to the scientist that has the hand on what filter he wants to enable for a specific data collect.
On the user side, a user can manage his privacy concerns by configuring constrain conditions in which experiments can collect data.
The methods implemented by the platform require that each part, the scientist and the user are aware of the risks of their privacy.
Moreover, existing privacy approaches are generally made on server side like k-Anonymity based methods~\cite{DBLP:journals/ijufks/Sweene02}~\cite{DBLP:conf/icde/LiLV07}~\cite{DBLP:conf/icde/MachanavajjhalaGKV06}.
Users don't want to put all their trust on servers by hoping, fingers crossed, that their data will be in a safe place and servers will always protect their privacy.
\\
% Basic idea/approach of the paper

The idea behind this contribution is to protect by hiding the data producers from potentially malicious servers by providing a solution that could scramble the data before sending it to the server.
The proposed approach uses a collaborative dissemination process where the users of the system exchange their data with each other before sending it to a collect server.
By this way, a server that receives data from a user will be unable to claim if these data are produced by the sender or not.
The final goal is to reach to a platform that could use Privacy by design-principles~\cite{langheinrich2001privacy} where, by conception a system could enhance the users' privacy without the need of more requirements.
\\
% Contributions of the paper

Our two contributions for crowd-sensing platforms are as follows:
\begin{itemize}
	\item Foug\`ere, a data dissemination library that uses a distance notion to disseminate data.
	The library comes with 3 defaults dissemination modules: Wi-Fi Direct, Social and Contextual.
	\item AndroFleet, a large-scale emulation platform that includes a WiFi-Direct emulation implementation for the Android platform. 
	This emulation platform is a tool that permits the testing of crowd-sensing applications in the context of a crowd of devices. 
	AndroFleet trends to be as near as possible to the real usage of the application on real mobile devices.
\end{itemize} 

% Textual contents description

The reminder of this paper is structured as follows.
In Section~\ref{sec:context_motivation} we introduce the context and motivations that are behind this work.
Section~\ref{sec:contribution} presents the principles used by the Foug\`ere library.
Section~\ref{sec:implementation_details} describes key issues in the Foug\`ere implementation.
Then, in the Section~\ref{sec:evaluation} we present AndroFleet and the results of the contribution.
Finally, in Section~\ref{sec:conclusion_perspectives} we summarize and discuss future possible works.

%%%% END OK
